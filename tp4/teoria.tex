\documentclass[12pt]{article}
\usepackage{amsmath}
\usepackage{amssymb}
\usepackage{graphicx}
\usepackage{color}
\usepackage{amsthm}

\newcommand{\proportionleft}[2]{\frac{#1}{#1 + #2}}
\newcommand{\proportionright}[2]{\frac{#2}{#1 + #2}}
\newcommand{\gaussian}[2]{\mathcal{N}\left(#1, #2\right)}

\begin{document}

Sea la matriz del Kalman Filter:
$$K = \proportionleft{\overline{\sigma}^2}{\sigma^2_{obs}}$$
y las fórmulas de corrección:
\begin{align*}
	\mu &= \overline{\mu} + K(z - \overline{\mu}) \\
	\sigma^2 &= (1 - K)\overline{\sigma}^2
\end{align*}
Demostraremos que la multiplicación de distribuciones normales:
$$\gaussian{\overline{\mu}}{\overline{\sigma}^2}\gaussian{z}{\sigma^2_{obs}}$$
Es proporcional a la distribución normal:
$$\gaussian{\mu}{\sigma^2}$$

\proof{
\begin{align*}
	\gaussian{\overline{\mu}}{\overline{\sigma}^2}\gaussian{z}{\sigma^2_{obs}} &\propto
	\gaussian{
		\proportionright{\overline{\sigma}^2}{\sigma^2_{obs}}\overline{\mu}+
		\proportionleft{\overline{\sigma}^2}{\sigma^2_{obs}}z
	}{
		\frac{1}{\overline{\sigma}^{-2} + \sigma^{-2}_{obs}}
	} \\
\end{align*}
Donde ahora demostramos por separado la igualdad de cada uno de los hiperparámetros de la distribución normal:
\begin{align*}
	\mu &= \overline{\mu} + K(z - \overline{\mu}) \\
	    &= \overline{\mu} + Kz - K\overline{\mu} \\
	    &= \overline{\mu} + \proportionleft{\overline{\sigma}^2}{\sigma^2_{obs}}z - \proportionleft{\overline{\sigma}^2}{\sigma^2_{obs}}\overline{\mu} \\
	    &= \left(1 - \proportionleft{\overline{\sigma}^2}{\sigma^2_{obs}}\right)\overline{\mu} + \proportionleft{\overline{\sigma}^2}{\sigma^2_{obs}}z \\
	    &= \proportionright{\overline{\sigma}^2}{\sigma^2_{obs}}\overline{\mu} + \proportionleft{\overline{\sigma}^2}{\sigma^2_{obs}}z \\
\end{align*}
\begin{align*}
	\sigma^2 &= (1 - K)\overline{\sigma}^2 \\
		 &= \left(1 - \proportionleft{\overline{\sigma}^2}{\sigma^2_{obs}}\right)\overline{\sigma}^2 \\
		 &= \proportionright{\overline{\sigma}^2}{\sigma^2_{obs}}\overline{\sigma}^2 \left(\frac{\sigma^{-2}_{obs}}{\sigma^{-2}_{obs}}\right) \left(\frac{\overline{\sigma}^{-2}}{\overline{\sigma}^{-2}}\right) \\
		 &= \frac{1}{\overline{\sigma}^{-2} + \sigma^{-2}_{obs}}
\end{align*}
\qed
}

\section*{Ejercicio 2}

Dadas las ecuaciones del modelo de movimiento $g$:
\begin{align*}
	x_{t+1} &= x_t + \delta_{trans} \cos(\theta_t + \delta_{rot1}) \\
	y_{t+1} &= y_t + \delta_{trans} \sin(\theta_t + \delta_{rot1}) \\
	\theta_{t+1} &= \theta_t + \delta_{rot1} + \delta_{rot2}
\end{align*}

Computamos los jacobianos de $g$ respecto del estado $G = \frac{\partial g}{\partial s}$ y del control $V = \frac{\partial g}{\partial u}$:

\begin{align*}
	G &= \begin{bmatrix}
		\frac{\partial x_{t+1}}{\partial x_t} & \frac{\partial x_{t+1}}{\partial y_t} & \frac{\partial x_{t+1}}{\partial \theta_t} \\
		\frac{\partial y_{t+1}}{\partial x_t} & \frac{\partial y_{t+1}}{\partial y_t} & \frac{\partial y_{t+1}}{\partial \theta_t} \\
		\frac{\partial \theta_{t+1}}{\partial x_t} & \frac{\partial \theta_{t+1}}{\partial y_t} & \frac{\partial \theta_{t+1}}{\partial \theta_t}
	\end{bmatrix} = \begin{bmatrix}
		1 & 0 & -\delta_{trans}\sin(\theta_t + \delta_{rot1}) \\
		0 & 1 & \delta_{trans}\cos(\theta_t + \delta_{rot1}) \\
		0 & 0 & 1
	\end{bmatrix} \\
	V &= \begin{bmatrix}
		\frac{\partial x_{t+1}}{\partial \delta_{rot1}} & \frac{\partial x_{t+1}}{\partial \delta_{trans}} & \frac{\partial x_{t+1}}{\partial \delta_{rot2}} \\
		\frac{\partial y_{t+1}}{\partial \delta_{rot1}} & \frac{\partial y_{t+1}}{\partial \delta_{trans}} & \frac{\partial y_{t+1}}{\partial \delta_{rot2}} \\
		\frac{\partial \theta_{t+1}}{\partial \delta_{rot1}} & \frac{\partial \theta_{t+1}}{\partial \delta_{trans}} & \frac{\partial \theta_{t+1}}{\partial \delta_{rot2}}
	\end{bmatrix} = \begin{bmatrix}
		-\delta_{trans}\sin(\theta_t + \delta_{rot1}) & \cos(\theta_t + \delta_{rot1}) & 0 \\
		\delta_{trans}\cos(\theta_t + \delta_{rot1}) & \sin(\theta_t + \delta_{rot1}) & 0 \\
		1 & 0 & 1
	\end{bmatrix}
\end{align*}

\end{document}

